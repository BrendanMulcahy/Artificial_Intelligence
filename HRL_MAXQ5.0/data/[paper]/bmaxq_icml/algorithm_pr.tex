\section{Bayesian MAXQ Algorithm}
\label{sec:algo}
\begin{algorithm}[t]
\caption{{\sc Bayesian\_maxq}} \label{alg:bmaxq}
\begin{algorithmic}[1]
\REQUIRE Task $i$, State $s$, Update Interval $k$, Simulation Episodes $Sim$
\ENSURE Next state $s'$, steps taken $N$
\IF{$i$ is primitive}
\STATE Execute $i$, observe $r$, $s'$
\STATE Update current posterior parameters $\Psi$ using ($s$, $i$, $r$, $s'$)~\label{line:update}
\STATE Update current value estimate: $V(i,s) \leftarrow (1-\alpha)\cdot V(i,s)+\alpha\cdot r$
\STATE $Count(i) \leftarrow Count(i)+1$
\RETURN $(s', 1, r)$
\ELSE 
\STATE $N \leftarrow 0$ \COMMENT{$i$ is composite}
\STATE $taskStack \leftarrow Stack()$, where taskStack is the stack of tuple $\langle task, state, cr \rangle$ %visited while executing each subtasks of i, $cr$ means cumulative rewards obtained from executing the task $a$, exiting from state $s$.
\STATE $CR \leftarrow 0$
\WHILE {$i$ is not terminated} 
\STATE {\sc Recompute\_value}$(i, k, Sim)$
\STATE $a\leftarrow \pi_{g}(i, s)$ \COMMENT{$\pi_{g}$ is the greedy policy from the current value function}
\STATE $\langle s', N_a, cr\rangle \leftarrow$ {\sc Bayesian\_maxq}($a$, $s$) \COMMENT{recursive call to child task}~\label{line:recursive}
\STATE $taskStack.push(\langle a, s', cr \rangle)$    
\STATE $a^*\leftarrow \arg\max_{a'}\bigl[C(i,s',a')+V(a',s')\bigr]$ \COMMENT{update completion function in $i$}~\label{line:comp1}
\STATE $C(i,s,a)\leftarrow(1-\alpha)\cdot C(i,s,a) + \alpha\cdot \gamma^{N_a}\bigl[C(i,s',a^*)+V(a^*,s') \bigr]$~\label{line:comp2}
\STATE $\tilde{C}(i,s,a)\leftarrow(1-\alpha)\cdot \tilde{C}(i,s,a) + \alpha\cdot \gamma^{N_a}\bigl[${\sc Pr}$(i, s')+\tilde{C}(i,s',a^*)+V(a^*,s') \bigr]$
\STATE $s\leftarrow s'$
\STATE $CR \leftarrow CR+cr$
\STATE $N\leftarrow N+N_a$
\STATE $Count(i) \leftarrow Count(i)+1$
\ENDWHILE
\STATE {\sc Update\_pseudo\_reward}($taskStack$)
\RETURN $(s', N, CR)$
\ENDIF
\end{algorithmic}
\end{algorithm} 

In this section, we describe our approach to incorporating
probabilistic priors into HRL. The intuition behind our approach is
quite straightforward: we follow the basic Bayesian model-based RL
procedure outlined in Section~\ref{sec:relwork} and adapt it to the
{\sc maxq} framework. At each step we have a distribution over
environment models (initially the prior). Our algorithm has two
subroutines: the main {\sc Bayesian\_MAXQ} routine
(Algorithm~\ref{alg:bmaxq}) and an auxiliary {\sc Recompute\_value}
routine (Algorithm~\ref{alg:recompvalue}). Intuitively, the main {\sc
  Bayesian\_MAXQ} routine interacts with the world using the current
value estimates and updates the posterior over the environment models.
Every $k$ steps, it calls the {\sc Recompute\_value} function. This
function then resamples a model from the posterior and recomputes the
value and completion functions to reflect this new model. Just as in
standard model-based RL, this procedure continues until the
hierarchical policy converges. In this description, the value $V$ and
completion $C$ functions are assumed to be global.

%%%%%%%%%%%%%%%%%%%%%%%%%%%%%%%%%%%
%%%% Update-pseudo-reward
%%%%%%%%%%%%%%%%%%%%%%%%%%%%%%%%%%%
\begin{algorithm}[t]
\caption{{\sc Update\_pseudo\_reward}}
\begin{algorithmic}[1]
\REQUIRE Stack$\langle Task, ExitState, CumulativeReward \rangle$ $taskStack$
\STATE $ tempCR \leftarrow 0$
\STATE $\langle a', s', cr' \rangle \leftarrow taskStack.pop()$
\WHILE {$taskStack$ is not empty}
\STATE $\langle a, s, cr \rangle \leftarrow taskStack.pop()$
\STATE $ tempCR \leftarrow tempCR+cr'$
\STATE $cr' \leftarrow cr$
\STATE Update current posterior parameters $\Phi$ using ($a$, $s$, $tempCR$)
\ENDWHILE
\end{algorithmic}
\end{algorithm}

%%%%%%%%%%%%%%%%%%%%%%%%%%%%%%%%%%%
%%%%  Compute-policy
%%%%%%%%%%%%%%%%%%%%%%%%%%%%%%%%%%%
\begin{algorithm}[t]
\caption{{\sc Recompute\_value}} \label{alg:recompvalue}
\begin{algorithmic}[1]
\REQUIRE Task $i$, Update Interval $k$, Simulation Episodes $Sim$
\ENSURE Recomputed value and completion functions for the task graph below and including $i$
\IF{$Count(i) < k$}
\RETURN~\label{line:checkk}
\ENDIF
\IF{$i$ is primitive}  
\STATE Sample new transition and reward parameters $\Theta$ from
current posterior $\Psi$~\label{line:sample}
\ELSE
\FORALL {child tasks $a$ of $i$}
\STATE {\sc Recompute\_value}($a$, $k$, $Sim$)
\ENDFOR
\FOR{$Sim$ episodes}
\STATE $s \leftarrow$ random nonterminal state of $i$
\STATE Run {\sc maxq-q}($i$, $s$, $\Theta$) \COMMENT{Table 4 in~\cite{d-hrl-00}}
\ENDFOR
\ENDIF
\STATE $Count(i) \leftarrow 0$
\end{algorithmic}
\end{algorithm} 


At the start of each episode, the {\sc Bayesian\_MAXQ} routine is
called with the $Root$ task and the initial state for the current
episode. The standard MAXQ execution protocol is then followed, where
each task chooses an action based on its current value function
(initially random). When a primitive action is reached and executed,
it updates the posterior over model parameters
(Line~\ref{line:update}) and its own value estimate (which is just the
reward function for primitive actions). When a task exits and returns
to its parent, the parent subsequently updates its completion function
based on the current estimates of the value of the exit state
(Lines~\ref{line:comp1}~and~\ref{line:comp2}). Note that in {\sc
  maxq}, the value function of a composite task can be (recursively)
computed using the completion functions of subtasks and the rewards
obtained by executing primitive actions, so we do not need to
separately store or update the value functions (except for the
primitive actions where the value function is the reward). Finally,
each primitive action maintains a count of how many times it has been
executed and each composite task maintains a count of how many child
actions have been taken.

When $k$ (an algorithm parameter) steps have been executed in a
composite task, {\sc Bayesian\_MAXQ} calls {\sc Recompute\_value} to
re-estimate the value and completion functions (the check on $k$ is
shown in {\sc Recompute\_value}, Line~\ref{line:checkk}). When
activated, this function recursively re-estimates the value/completion functions for all
 subtasks of the current task. At the level of a primitive action, this
simply involves resampling the reward and transition parameters from
the current posterior over models. For a composite task, we use the
{\sc maxq-0} algorithm (Table 2 in~\cite{d-hrl-00}).~\footnote{The {\sc
    maxq-0} algorithm assumes there are no {\em pseudo-rewards}, i.e.,
  preferences over possible exits for a subtask. If this is not true,
  this can be replaced by the {\sc maxq-q} algorithm.} We run this algorithm
for $Sim$ episodes, starting with the current subtask as the root. This algorithm
recursively updates the completion function of the task graph below
the current task. Note that in this step, the subtasks with primitive
actions use model-based updates. That is, when a primitive action is
``executed'' in such tasks, the currently sampled transition function
(part of $\Theta$ in Line~\ref{line:sample}) is used to find the next
state, and then the associated reward is used to update the completion
function. This is similar to
Lines~\ref{line:recursive},~\ref{line:comp1} and \ref{line:comp2} in
{\sc Bayesian\_maxq}, except that it uses the sampled model $\Theta$
instead of the real environment. After {\sc Recompute\_value}
terminates, a new set of value/completion functions are available for
{\sc Bayesian\_maxq} to use to select actions.

We note that in the ideal case, instead of running for $Sim$ episodes,
we would simply run {\sc maxq-0} to convergence; in practice this is
not possible, so we impose an arbitrary cutoff. Similarly, ideally we
would set $k=1$. In this case, in fact, Lines~\ref{line:comp1}
and~\ref{line:comp2} are not needed, since the completion functions
would be recomputed every step. Again, this is generally not possible in
practice. 

